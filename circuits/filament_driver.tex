\documentclass[margin = 24mm]{standalone}

\usepackage{mathspec}
\setallmainfonts(Digits,Latin){osifont.ttf}

\usepackage{siunitx}
\sisetup{locale = DE, retain-explicit-plus}

\usepackage[european]{circuitikz}

\begin{document}
	\begin{circuitikz}[]
		% left pnp
		\node [pnp, xscale = -1] (Q1) at (2, -3) {\ctikzflipx{$Q_1$}};
		\draw (2, 0) to [R = $R_1$] (Q1.E);

		% resistor to ground
		\draw (Q1.C) to [R = $R_2$] ++(0, -2) to node[rground] {} ++(0, 0);

		% right pnp
		\node [pnp] (Q2) at (4.5, -3) {$Q_2$};
		\path (4.5, 0) -- (Q2.E) coordinate [pos = .5] (dmid);
		\draw (4.5, 0) to [empty diode] (dmid) to [empty diode] (Q2.E);

		% resistor to ground
		\draw (Q2.C) to [R = $R_3$] ++(0, -2) to node[rground] {} ++(0, 0);

		% connect the first collector to the bases
		\path (Q1.C) -- ++(0, -2) coordinate [pos = 0] (bceout);
		\draw (Q1.B) -- (Q2.B) coordinate [pos = .2] (bcein);
		\draw (bceout) -| (bcein);
		\path (bceout) to [short, *-*] (bcein);

		% third resistor, this one's carrying all the load:
		\path (Q2.C) -- ++(3, 0) coordinate [pos = 1] (q2pos);
		\node [npn, scale = 1.3] (Q3) at (q2pos) {$Q_3$};
		\draw (Q2.C) to [short, *-] (Q3.B);

		% connect everything to +5V:
		\node [above] at (0, 0) { \SI{+5}{\volt} };
		\draw (0, 0) to [short, o-*] (2, 0) to [short, -*] (4.5, 0) -| (Q3.C);

		% output node:
		\draw (Q3.E) to [short, -o] ++(1, 0);
		\path (Q3.E) -- ++(1, 0) node[pos = 1, above] {$V_{out}$};

	\end{circuitikz}
\end{document}
